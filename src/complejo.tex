\documentclass{article}

% codificacion
%\usepackage[T1]{fontenc}
\usepackage[utf8]{inputenc}
\usepackage[spanish,es-tabla]{babel}

% modifica sangria
\parindent=0cm

% expresiones matematicas
\usepackage{amsmath}
\usepackage{amssymb}
\usepackage{amsfonts}
\usepackage{latexsym}
\usepackage{cancel}
\usepackage{bm}        % texto en bold

% manejar e incluir graficas
\usepackage{graphicx}

% conversiona formato eps a pdf
%\usepackage{epstopdf}

% formato extra en figuaras o tablas (H)
\usepackage{float}

% permite colocar varias figuras en el entorno figure
\usepackage{subfigure}

% formatos para las tablas
\usepackage{array}
\usepackage{longtable}

% nuevo formato para las tablas
\newcolumntype{E}{>{$}c<{$}}

% restricción para las matrices
\setcounter{MaxMatrixCols}{40}

% manejar los margenes
\usepackage{geometry}
\usepackage{fancyhdr}

\usepackage{lipsum}
%\usepackage{showframe}

% manejo de columnas
\usepackage{multicol}

% manejo de secciones
\usepackage{titling}
\usepackage{titlesec}

\usepackage[sc]{mathpazo}

% para hipervinculos
\usepackage[colorlinks = true,
            linkcolor = black,
            urlcolor = blue,
            citecolor = blue]{hyperref}

% <<<<<<<<<<<<<< SECCIONES >>>>>>>>>>>>>>>
	% cambio el formato de numeración de las secciones
	\renewcommand{\thesection}{\Roman{section}}
	\renewcommand{\thesubsection}{\roman{subsection}}
	% centro las secciones
	\titleformat{\section}
	    [block]                     % ayuda a poder centrar
	    {\bfseries\centering}       % estilo
	    {\thesection.}              % contenido
	    {1mm}{}                     % margen de numeración
	\titleformat{\subsection}[block]{\bfseries\centering}{\thesubsection.}{1mm}{}

% <<<<<<<<<<<<<< ENCABEZADO Y PIE >>>>>>>>>>>>>>>
    % uso de encabezado y pie
    \pagestyle{fancy}
    % limpio formatos anteriores
	\fancyfoot{}\fancyhead{}
	% nuevo formato
	\fancyhead[L]{
	    \textbf{PROVINCIA DE NEUQUÉN\\
		CONSEJO PROVINCIAL DE EDUCACIÓN\\
		E.P.E.T. N$^\circ$ 20 NEUQUÉN}
	}
	%\fancyhead[C]{Titulo}
	\fancyhead[R]{\textbf{Lanin 2020\\Tel. 4478052}}
	%\fancyhead[R]{imagen}
	\renewcommand{\headrulewidth}{0.9pt}
	
	\fancyfoot[R]{\thepage}
	\fancyfoot[L]{Bruno Quiroga, Isaac}
	\renewcommand{\footrulewidth}{0.5pt}

% <<<<<<<<<<<<<< GEOMETRY >>>>>>>>>>>>>>>
	\geometry{
	    left=3cm,
	    width=19cm,
	    top=1cm,
	    bottom=2cm,
	    height=22cm,
	    headsep=6mm,
	    head=1.5cm,
	    foot=1cm,
	    marginparsep=5mm,
	    marginpar=3cm,
	    includeall,
	}

% <<<<<<<<<<<<<< SETEO DE MAKETITLE >>>>>>>>>>>>>>>
	% lleva más arribla el titulo
	\setlength{\droptitle}{-5\baselineskip}
	\title{{\Huge\textbf{Titulo del articulo}}}
	\author{
	    \textsc{Isaac Bruno Quiroga}
		\thanks{Información relacionada con el autor}\\[0.2cm]
		\normalsize Universidadd de Guadalajara \\
		\normalsize
		\href{mailto:isbrqu@gmail.com}{isbrqu@gmail.com}
	}

\begin{document}

    \begin{titlepage}
        \begin{center}
	        % \vspace*{2\baselineskip}
	        \hrule height 3pt
	        \vspace*{0.5\baselineskip}
	        {\Huge \textbf{E.P.E.T. N$^\circ$ 20}}
	        \\[0.1cm]
	        {\large \textbf{Ciclo superior: 5$^{to}$ 2$^{da}$}}
	        \vspace*{0.5\baselineskip}
	        \hrule
	        \vspace*{7\baselineskip}
	        \includegraphics[scale=1]{example-image-a}
	        \vspace*{2\baselineskip}
	        \\
	        {\huge\textbf{Técnicas Avanzadas en Programación}}
	        \vfill
	        {\Large Bruno Quiroga Isaac}\\
	        \today
        \end{center}
    \end{titlepage}
    
    % <<<<<<<<<<<<<< INDEX >>>>>>>>>>>>>>>
	    % renombra el titulo del indice
	    \renewcommand{\contentsname}{Contenido}
	    \tableofcontents
	    % quitar el formato de la pagina del pie y encabezado
	    \thispagestyle{empty}

        \newpage
    
    % <<<<<<<<<<<<<< CONTENIDO >>>>>>>>>>>>>>>
    % para que empiece de la uno la primera pagina y no como la segunda
    \setcounter{page}{1}
    
    \maketitle
    
    \begin{abstract}
        \lipsum[7]
    \end{abstract}

    \begin{multicols}{2}
	    \section{Introducción}
	    
	    \lipsum[1]. Como la figura \eqref{figuara1} ejemplifica.
	    
	    Este es otro texto que pretende ejemplificar algo escrito a medida que lo voy haciente \footnote{Hablo de un ejemplo bien chetooo o una explicación agregada}
	    
	    \begin{figure*}[ht]%[hb]
	        \centering
	        \includegraphics[scale=0.4]{example-image-c}
	        \caption{Primera figura}
	    \end{figure*}
	    
	    \lipsum[3]
	    
	    \subsection{Bases}
	    
	    \lipsum[3]. Como en la tabla \eqref{mathtable} ejemplifica.
	    
	    
	    \lipsum[4]
	    
	    \subsection{Ejemplos}
	    
	    \lipsum[6]\cite{nombre1}
	    
	    \begin{table}[H]
	        \centering
	        \begin{tabular}{|c|c|c|}
	            \hline
	            elemento uno  & elemento dos & elemento tres \\
	            \hline
	            elemento uno  & elemento dos & elemento tres \\
	            \hline
	            elemento uno  & elemento dos & elemento tres \\
	            \hline
	        \end{tabular}
	        \caption{Usando el entorno normal}
	        \label{tabla1}
	    \end{table}
	    
	    \subsection{Productos notables}
	    
	    \lipsum[1-2]
	    
	    \begin{figure}[H]
	        \centering
	        \includegraphics[scale=0.5]{example-image-b}
	        \caption{Gráfica importante}
	        \label{figuara1}
	    \end{figure}
	    
	    \lipsum[1-2] \cite{nombre2}
	    
	    \section{Cálculo diferencial}
	    
	    \lipsum[1] \cite{nombre1}	    
	    
	    \begin{table}[H]
	        \centering
	        \begin{tabular}{|E|E|E|}
	            \hline
	            x & (x+1)^2 & (x+1)^3 \\
	            \hline
	            x & (x+1)^2 & (x+1)^3 \\
	            \hline
	            x & (x+1)^2 & (x+1)^3 \\
	            \hline
	        \end{tabular}
	        \caption{Usando el entorno matemático definido}
	        \label{mathtable}
	    \end{table}
	    
	    \subsection{Más ejemplos}
	    
	    \lipsum[3]
	    
	    \subsection{Otros ejemplos más complicados}
	    
	    \lipsum[2]
	    
        \begin{figure*}[ht]
	        \centering
	        \subfigure[uno]{\includegraphics[scale=0.3]{example-image-a}}
	        \hspace{0.3cm}
	        \subfigure[dos]{\includegraphics[scale=0.3]{example-image-b}}
	        \caption{Graficas multiples}
	    \end{figure*}
	    
	    \lipsum[1]
	    
	    \lipsum[3]

	    \lipsum[1]
	    
	    \lipsum[3]
	    
	    \begin{table*}[ht]
	        \centering
	        \begin{tabular}{|c|c|c|}
	            \hline
	            elemento uno  & elemento dos & elemento tres \\
	            \hline
	            elemento uno  & elemento dos & elemento tres \\
	            \hline
	            elemento uno  & elemento dos & elemento tres \\
	            \hline
	        \end{tabular}
	        \caption{Usando el entorno normal}
	        \label{tabla2}
	    \end{table*}
	    
	    \lipsum[1]
	    
	    \lipsum[3]
	    
	    \section{Cosas interesantes}
	    
	    \lipsum[7]
	    
	    \lipsum[2]

	    \lipsum[2]
	    
	    \lipsum[2]
	    
	    \lipsum[2]
	    
	    \section{Más y más y más}
	    
	    \lipsum[7]
	    
	    \lipsum[7]
	    
	    \lipsum[7]
	    
	    \lipsum[7]
	    
	    \lipsum[2]
	
    \end{multicols}
    
    % <<<<<<<<<<<<<< BIBLIOGRAFIA >>>>>>>>>>>>>>>
	    \newpage
	    \renewcommand{\refname}{Bibliografía} % cambia el nombre de referencia por bibliografia
	    \begin{thebibliography}{99}
	        \bibitem{nombre1} Articulo o Libro 1, Autor 1, año de referencia 2
	        \bibitem{nombre2} Articulo o Libro 2, Autor 2, año de referencia 2
	        \bibitem{nombre3} Articulo o Libro 3, Autor 3, año de referencia 3
	    \end{thebibliography}
    
\end{document}
