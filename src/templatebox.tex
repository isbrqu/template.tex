\documentclass{article}

\usepackage[most]{tcolorbox}
\usepackage{lipsum}
\usepackage{xcolor}
\usepackage[utf8]{inputenc}
\usepackage[T1]{fontenc} 
\usepackage{lmodern}
\usepackage[spanish]{babel}
\usepackage{fontawesome}

\tcbsetforeverylayer{
    colframe=black,
    colback=white,
    colbacktitle=white,
    coltitle=black
}

\tcbset{
    fonttitle=\bfseries\upshape
}

\newtcolorbox{tagbox}[2][]{
    enhanced, hbox,
    before skip=2mm,after skip=1mm,
    boxrule=0.4pt,left=5mm,right=4mm,top=1mm,bottom=1mm,
    sharp corners,rounded corners=southeast,arc is angular,arc=3mm,
    underlay={
        \path[fill=black]
            ([yshift=3mm]interior.south east)--++(-0.4,-0.1)--++(0.1,-0.2);
        \path[draw=black,shorten <=-0.05mm,shorten >=-0.05mm]
            ([yshift=3mm]interior.south east)--++(-0.4,-0.1)--++(0.1,-0.2);
        \path[fill=black,draw=none]
            (interior.south west) rectangle node[white]{\Large\bfseries #2} ([xshift=4mm]interior.north west);
    },
    drop fuzzy shadow,#1
}

\newtcolorbox{questionbox}[1][]{
    enhanced,
    before skip=2mm,after skip=1mm,
    boxrule=0.4pt,left=5mm,right=4mm,top=1mm,bottom=1mm,
    sharp corners,rounded corners=southeast,arc is angular,arc=3mm,
    underlay={
        \path[fill=black]
            ([yshift=3mm]interior.south east)--++(-0.4,-0.1)--++(0.1,-0.2);
        \path[draw=black,shorten <=-0.05mm,shorten >=-0.05mm]
            ([yshift=3mm]interior.south east)--++(-0.4,-0.1)--++(0.1,-0.2);
        \path[fill=black,draw=none]
            (interior.south west) rectangle node[white]{\huge\bfseries ?} ([xshift=4mm]interior.north west);
    },
    drop fuzzy shadow, #1
}

\newtcolorbox{importantbox}[1][]{
    enhanced,
    before skip=2mm,after skip=1mm,
    boxrule=0.4pt,left=5mm,right=4mm,top=1mm,bottom=1mm,
    sharp corners,rounded corners=southeast,arc is angular,arc=3mm,
    underlay={
        \path[fill=black]
            ([yshift=3mm]interior.south east)--++(-0.4,-0.1)--++(0.1,-0.2);
        \path[draw=black,shorten <=-0.05mm,shorten >=-0.05mm]
            ([yshift=3mm]interior.south east)--++(-0.4,-0.1)--++(0.1,-0.2);
        \path[fill=black,draw=none]
            (interior.south west) rectangle node[white]{\huge\bfseries !} ([xshift=4mm]interior.north west);
    },
    drop fuzzy shadow, #1
}

\newtcolorbox{remarkbox}[1][]{
    enhanced,
    before skip=2mm,after skip=3mm,
    boxrule=0.4pt,left=6mm,right=4mm,top=1mm,bottom=1mm,
    sharp corners,rounded corners=southeast,arc is angular,arc=3mm,
    underlay={
        \path[fill=black]
            ([yshift=3mm]interior.south east)--++(-0.4,-0.1)--++(0.1,-0.2);
        \path[draw=black,shorten <=-0.05mm,shorten >=-0.05mm]
            ([yshift=3mm]interior.south east)--++(-0.4,-0.1)--++(0.1,-0.2);
        \path[fill=black,draw=black, line width=.4pt]
            (interior.south west) rectangle node[white]{\faEye} ([xshift=5mm]interior.north west);
    },
    drop fuzzy shadow,#1
}

\newtcolorbox{exercisebox}[1][]{
    enhanced,
    before skip=2mm,after skip=3mm,
    boxrule=0.4pt,left=6mm,right=4mm,top=1mm,bottom=1mm,
    sharp corners,rounded corners=southeast,arc is angular,arc=3mm,
    underlay={
        \path[fill=black]
            ([yshift=3mm]interior.south east)--++(-0.4,-0.1)--++(0.1,-0.2);
        \path[draw=black,shorten <=-0.05mm,shorten >=-0.05mm]
            ([yshift=3mm]interior.south east)--++(-0.4,-0.1)--++(0.1,-0.2);
        \path[fill=black,draw=black, line width=.4pt]
            (interior.south west) rectangle node[white]{\faPencil} ([xshift=5mm]interior.north west);
    },
    drop fuzzy shadow,#1
}

\newtcolorbox{exercisepbox}[1][]{
    enhanced,
    before skip=2mm,after skip=3mm,
    boxrule=0.4pt,left=6mm,right=4mm,top=1mm,bottom=1mm,
    sharp corners,rounded corners=southeast,arc is angular,arc=3mm,
    underlay={
        \path[fill=black]
            ([yshift=3mm]interior.south east)--++(-0.4,-0.1)--++(0.1,-0.2);
        \path[draw=black,shorten <=-0.05mm,shorten >=-0.05mm]
            ([yshift=3mm]interior.south east)--++(-0.4,-0.1)--++(0.1,-0.2);
        \path[fill=black,draw=black, line width=.4pt]
            (interior.south west) rectangle node[white]{\faEdit} ([xshift=5mm]interior.north west);
    },
    drop fuzzy shadow,#1
}

\newtcolorbox{blanckbox}[2][]{
    enhanced, hbox,
    before skip=2mm,after skip=3mm,
    boxrule=0.4pt,left=8mm,right=5mm,top=1mm,bottom=1mm,
    sharp corners,rounded corners=southeast,arc is angular,arc=3mm,
    underlay={
        \path[fill=black] ([yshift=3mm]interior.south east)--++(-0.4,-0.1)--++(0.1,-0.2);
        
        \path[draw=black,shorten <=-0.05mm,shorten >=-0.05mm] ([yshift=3mm]interior.south east)--++(-0.4,-0.1)--++(0.1,-0.2);
        
        \path[fill=white,draw=black, line width=.4pt] (interior.south west) rectangle node[black]{\Large\bfseries #2} ([xshift=7mm]interior.north west);
    },
    drop fuzzy shadow,#1
}

\newtcolorbox{leftrightbox}[3][]{
    enhanced, hbox,
    before skip=2mm,after skip=3mm,
    boxrule=0.4pt,left=7mm,right=5mm,top=1mm,bottom=1mm,
    sharp corners,
    underlay={
        \path[fill=black,draw=none] (interior.south west) rectangle node[white]{\Large\bfseries #2} ([xshift=6mm]interior.north west);
        
        \path[fill=black,draw=none] (interior.south east) rectangle node[white]{\Large\bfseries #3}([xshift=-4mm]interior.north east);
    },
    drop fuzzy shadow,#1
}

\begin{document}

    \begin{questionbox}
        ¿Comó se puede vivir mejor?
    \end{questionbox}
    
    \begin{importantbox}
        Es esencial lo que dice aquí
    \end{importantbox}
    
    \begin{remarkbox}
        Este cartel le tenes que prestar atención
    \end{remarkbox}
    
    \begin{exercisebox}
        Comprobar si esto se puede hacer
    \end{exercisebox}
    
    \begin{exercisepbox}
        Comprobar si esto otro se puede hacer
    \end{exercisepbox}
    
    \begin{blanckbox}{\faEye}
        observación de cada punto
    \end{blanckbox}
    
    \begin{leftrightbox}[]{\faEdit}{?}
        habia una vez un pasto verde
    \end{leftrightbox}
    
    \newpage
    
    \begin{tcolorbox}[blanker,left=3mm,right=3mm,
        borderline vertical={2pt}{0pt}{black}]
        This is a \textbf{tcolorbox}.\\
        My second line.
    \end{tcolorbox}
    
    \begin{tcolorbox}[blanker,top=3mm,bottom=3mm,
        borderline horizontal={2pt}{0pt}{black}]
        This is a \textbf{tcolorbox}.
    \end{tcolorbox}

    \begin{tcolorbox}[
        enhanced,arc=3mm,boxrule=1.5mm,boxsep=1.5mm,
        colframe=black,
        borderline={1mm}{1mm}{white},
        borderline={1mm}{2mm}{black} ]
        \lipsum[1]
    \end{tcolorbox}
    
    \newtcolorbox{mybox}[2][]{title={#2},#1}
    
    \begin{mybox}{My title}
        This is a \textbf{tcolorbox}.
    \end{mybox}
    
    \begin{tcolorbox}[rightrule=3mm,title=myTitle]
        This is a \textbf{tcolorbox}.
    \end{tcolorbox}
    
    \begin{tcolorbox}[leftrule=3mm,title=myTitle]
        This is a \textbf{tcolorbox}.
    \end{tcolorbox}
    
    
    \begin{mybox}[detach title,before upper={\tcbtitle\quad}, leftrule=3mm]{My title}
        This is a \textbf{tcolorbox}.
    \end{mybox}
    
    \begin{mybox}[detach title,after upper={\par\hfill\tcbtitle}]{My title}
        This is a \textbf{tcolorbox}.
    \end{mybox}
    
    Some text\dotfill
    \tcbox[nobeforeafter,nobeforeafter,box align=base]{One line}
    \tcbox[nobeforeafter,nobeforeafter,box align=base,size=fbox]{Another line}
    
    \begin{tcolorbox}[before lower*=$,after lower*=$]
        This is a \textbf{tcolorbox}.
        \tcblower
        \sin^2(x)+\cos^2(x)=1.
    \end{tcolorbox}
    
    \newpage
    
    \begin{tcolorbox}[before title={\textcolor{black}{\large Important:}~},title=My title]
        This is a \textbf{tcolorbox}.
    \end{tcolorbox}
    
    \begin{tcolorbox}[after title={\hfill\colorbox{black}{\textcolor{white}{approved}}},title=My title]
        This is a \textbf{tcolorbox}.
    \end{tcolorbox}

    \begin{tcolorbox}[after upper={\par\hfill\textit{Read more next week}},title=My title]
        This is a \textbf{tcolorbox}.
    \end{tcolorbox}

    \begin{tcolorbox}[before upper=\flqq,after upper=\frqq]
        This is a \textbf{tcolorbox}.
    \end{tcolorbox}
    
    \begin{tcolorbox}[title=Box \thetcolorboxnumber]
        This box is \thetcolorboxnumber.
        \tcbox[on line,size=fbox]{This box is \thetcolorboxnumber} and
        \tcbox[on line,size=fbox]{this box is \thetcolorboxnumber}.
        This box is \thetcolorboxnumber.
    \end{tcolorbox}
    
    \newtcbtheorem[number within=section]{mytheo}{My Theorem}{}{th}
    
    \begin{mytheo*}{Unnumbered Theorem}
        This theorem is not numbered.
    \end{mytheo*}
    
    \begin{mytheo*}{}
        This theorem has no number and no title.
    \end{mytheo*}
    
    \newtcbtheorem[use counter from=mytheo]{theorem}{Teorema}{
    fonttitle=\bfseries\upshape,fontupper=\itshape,
    colframe=black,colback=white,
    colbacktitle=black,coltitle=black}{theo}
    
    \begin{theorem*}[theorem style=plain]{pomping}{}
        This is my theorem. \begin{equation*} a^2 + b^2 = c^2. \end{equation*}
        simplooon $1=2-1=1+1-1$
    \end{theorem*}
    
    \newenvironment{itemizebox}{\begin{itemize}}{\end{itemize}}
    \tcolorboxenvironment{itemizebox}
    {blanker, before skip=6pt,after skip=6pt, borderline west={3mm}{0pt}{red}}
    Some text.
    \begin{itemizebox}
        \item Alpha
        \item Beta
        \item Gamma
    \end{itemizebox}
    More text.

\end{document}