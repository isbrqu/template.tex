\documentclass{article}
\usepackage[utf8]{inputenc}
\usepackage[spanish]{babel}
\usepackage{amsmath}
\usepackage{lipsum}
\usepackage{multicol}
\usepackage{paracol}
\usepackage{graphicx}
\usepackage{xcolor}
\title{Titulo de documento}
\author{Isaac, Bruno Quiroga}
\date{\today}

\begin{document}
    \maketitle
    \begin{abstract}
        \lipsum[1]
    \end{abstract}
%    \section{Sección de verdades}
%        \lipsum[2]
%        \subsection{Verdades masivas}
%            \lipsum[1]
%        \subsubsection{Otras cosaaaas}
%            \lipsum[2]
%        \paragraph{Otro titulo}
%            \lipsum[7]

%    columnado feo
%    \twocolumn
%    \lipsum[1-3]
%    \onecolumn
%    \lipsum[1]

%   columnado maso maso (multicol}
%    \renewcommand{\columnseprule}{.5pt}
%    \setlength{\columnsep}{9pt}
%    \begin{multicols}{3}
%        \lipsum[1-5]
%    \end{multicols}
    \pagebreak
    
    \section{Sección de titulo largooo}
    
    \begin{paracol}{2}
        \lipsum[1]
        \switchcolumn
        Código fuente
        \switchcolumn*
        \lipsum[3]
        \switchcolumn
        \lipsum[4]
    \end{paracol}
    
    \columnratio{0.65}
    \setlength{\columnsep}{4em}
    \setlength{\columnseprule}{0.4pt}
    \begin{paracol}{2}
        column 1:
        \lipsum[1]
        \switchcolumn
        column 2:
        \lipsum[2]
    \end{paracol}
    
    \section{Columnas dispares}
    \columnratio{.6}
    \begin{paracol}{2}
        \lipsum[5]
        \switchcolumn
        \lipsum[6]
        \textcolor{green}{\lipsum[4]}
        \switchcolumn*
        \lipsum[1]
    \end{paracol}
    
    % \pagebreakellipse 
    
    \columnratio{.6} % Modifica el ancho de la columna 1 (primera a la izquierda)
    \setlength{\columnsep}{4em}
    \setlength{\columnseprule}{0.4pt}
	\begin{paracol}{2}[\section{Título muy largo para la primera sección}]
		\subsection{Columnas sincronizadas}
		\lipsum[3]
		\switchcolumn % Cambio de columna
		\lipsum[2]
		\switchcolumn* % Columna sincronizada
		Columna sincronizada
		\[
			\frac{d}{dx}(\sinh ^{-1}x) = \frac{1}{\sqrt{1 + x^2}}
		\]
		\switchcolumn
		\lipsum[2]
		\switchcolumn*
		\lipsum[1]
		\switchcolumn
		\lipsum[2]
    \end{paracol}

    \section{De vuelta a la distribución de una sola columna}
		\lipsum[7-8]
		\columnratio{.4}
		\columncolor{red} % cambia el color del texto de toda la columna
		\begin{paracol}{2}
			\lipsum[5]
			\switchcolumn
			\begin{figure}[ht] % Inserta una figura
				\centering
				\includegraphics[scale=.35]{example-image-a}
				\caption{Flotantes en paracol}
			\end{figure}
    \end{paracol}
    
\end{document}
