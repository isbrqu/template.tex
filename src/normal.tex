\documentclass{article}

% modifica sangria
\parindent=0cm

% codificacion
%\usepackage[T1]{fontenc}
\usepackage[utf8]{inputenc}
\usepackage[spanish,es-tabla]{babel}

% tester
\usepackage{showframe}
\usepackage{lipsum}

% margenes, cajas y hoja
\usepackage{geometry}

% imagen
\usepackage{graphicx}

% encabezado y pie 
\usepackage{fancyhdr}

% <<<<<<<<<<<<<< ENCABEZADO Y PIE >>>>>>>>>>>>>>>
    % uso de encabezado y pie
    \pagestyle{fancy}
    % limpio formatos anteriores
	\fancyfoot{}\fancyhead{}
	% nuevo formato
	\fancyhead[L]{
	    \textbf{PROVINCIA DE NEUQUÉN\\
		CONSEJO PROVINCIAL DE EDUCACIÓN\\
		E.P.E.T. N$^\circ$ 20 NEUQUÉN}
	}
	%\fancyhead[C]{Titulo}
	\fancyhead[R]{\textbf{Lanin 2020\\Tel. 4478052}}
	%\fancyhead[R]{imagen}

	\fancyfoot[R]{\thepage}
	\fancyfoot[L]{Bruno Quiroga, Isaac}

% <<<<<<<<<<<<<< GEOMETRY >>>>>>>>>>>>>>>
	\geometry{
		%left=3cm,
		%width=18.5cm,
		%top=1cm,
		%bottom=2cm,
		%height=18cm,
        headsep=18pt,
		headheight=1.5cm,
		%foot=1cm,
		%marginparsep=5mm,
		%marginpar=3cm,
		%includeall,
	}
	
% <<<<<<<<<<<<<< ENVIROMENT >>>>>>>>>>>>>>>
    
    

% <<<<<<<<<<<<<< DOCUMENTO >>>>>>>>>>>>>>>
\begin{document}

    % <<<<<<<<<<<<<< PORTADA >>>>>>>>>>>>>>>
	    \newgeometry{
	    	%left=3cm,
			%width=18.5cm,
			%top=1cm,
			%bottom=2cm,
		    height=22cm,
		    headheight=0cm,
			%headsep=20pt,
			%head=1.5cm,
			%foot=1cm,
			%marginparsep=5mm,
		    marginparwidth=0cm,
			%includeall,
	    }
	
	\begin{titlepage}
	    \begin{center}
		    %\vspace*{2\baselineskip}
		    \hrule height 3pt
		    \vspace*{0.5\baselineskip}
		    {\Huge \textbf{E.P.E.T. N$^\circ$ 20}}
		    \\[0.1cm]
		    {\large \textbf{Ciclo superior: 5$^{to}$ 2$^{da}$}}
		    \vspace*{0.5\baselineskip}
		    \hrule
		    \vspace*{7\baselineskip}
		    \includegraphics[scale=1]{example-image-a}
		    \vspace*{2\baselineskip}
		    \\
		    {\huge\textbf{Técnicas Avanzadas en Programación}}
		    \vfill
		    {\Large Bruno Quiroga Isaac}\\
		    \today
	    \end{center}
	\end{titlepage}
	
	\restoregeometry
    
    % <<<<<<<<<<<<<< INDEX >>>>>>>>>>>>>>>
	    % renombra el titulo del indice
	    \renewcommand{\contentsname}{Contenido}
	    \tableofcontents
	    % quitar el formato de la pagina del pie y encabezado
	    \thispagestyle{empty}

    % <<<<<<<<<<<<<< CONTENIDO >>>>>>>>>>>>>>>
    
    % escribir en el margen de notas
    \marginpar{This is a margin note using the latex}
    %\marginnote{This is a margin note using the geometry package, set at 3cm vertical offset to the line it is typeseted.}[3]
    
    % RESUMEN
    \begin{abstract}
        \lipsum[1]
    \end{abstract}
    
    \section{Introducción}
    
    \lipsum[2]
    
    \section{Dinámica de fluidos}
    
    \lipsum[3]
    
    \subsection{Producción masiva}
    
    \lipsum[4]
    
    \lipsum[5]

    \subsection{Explotación de animales}
    
    \lipsum[5]
    
    \lipsum[1]
    
    \section{Carbón}
    
    \lipsum[6]
    
    \lipsum[3]
    
    \subsection{Mecánicas}
    
    \lipsum[2]
    
    \subsection{Contaminación}
    
    \lipsum[3]
    
    \subsubsection{En el aire}
    
    \lipsum[6]
    
    \subsubsection{En el agua}
    
    \lipsum[3]
    
    \section{Energizantes naturales}
    
    \paragraph{Pasto}
    
    \lipsum[1]
    
    \paragraph{Hierro}
    
    \lipsum[2]
    
    \paragraph{Ejemplo}
    
    \lipsum[3]
        
\end{document}
